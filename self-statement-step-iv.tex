\documentclass[a4paper,11pt]{extarticle}
\usepackage[utf8]{inputenc}
\usepackage{geometry}
    \geometry{a4paper, margin = 0.82in}
\title{Self Statement}
\date{\vspace{-5ex}}

\usepackage[colorlinks=true, urlcolor=blue]{hyperref}

\begin{document}
\maketitle

\section*{Research}
Since my last review (Merit to Professor, Step III o/s, Effective July 1, 2021), I have pursued active research agendas across two main areas: Social and cultural theory and social network analysis.

\subsection*{Social and Cultural Theory}
My recent social and cultural theory work has split into three distinct lines, two of which will culminate in book-length treatments.

\subsubsection*{Cognitive Neuroscience and Social Theory}
For this review period, two papers (co-authored with Seth Abrutyn) highlight this line of work (\textbf{A143} and \textbf{A168}). \textbf{A143} deals with the relevance of recent work in affective neuroscience and the neurobiology of memory and imitation for theorizing the emotional roots of the social self. We use this work to move sociological theorizing on the self beyond its ``cognitivist'' bias (e.g., the self as a bundle of thoughts) and towards an ``emotional'' conception rooted in fundamental affective systems. We also theorize the temporal and interactional components of the self, noting how ``prospective'' aspects of the self (those used to look forward and imagine future possibilities) draw on the same neural resources used to construct images of the past. In both cases, the self is \textit{projected} to a constructed space, in the past, future, or even a hypothetical event. We contrast this \textit{projected self} to the \textit{embodied self} revealed by work on the ``mirror neuron'' system. This self is projected into a concrete interactional space (not a projected future or past space), defined by the perceived actions of embodied others and our attunement to those actions via motor resonance and embodied understanding.  

\textbf{A168} By way of contrast, draws on recent work on the neuroscience of motivation and drive, to try to breathe new life into the old sociological concept of \textit{role}. After noting the neglect of the role concept in recent theorizing on action, interaction, and institutions, we show that what is lacking is a proper conceptualization of the \textit{motivation} people have to actually engage in role-based behavior. We show that contemporary dramaturgical understandings of the role concept are deficient in this respect and recuperate an older idea from the functionalist tradition: That role-based action links to institutions via structure \textit{reward systems} and that rewards linked naturally to contemporary understandings of the neural bases of motivation.

\subsubsection*{Classical Social Theory and the ``Probabilism Project''}
This covers various projects concerned with both theoretical exegesis---mainly focused on the work of Max Weber and Pierre Bourdieu (\textbf{A159} and \textbf{A160})---and the development of an entirely new approach to social theory and social analysis labeled ``sociological probabilism.'' This line of work is also concerned with an editorial project dedicated to rethinking classical theory together with Seth Abrutyn, as exemplified in Springer's recently co-edited Handbook of Classical Social Theory (\textbf{A152}). Nevertheless, the probabilism project (together with Michael Strand of Brandeis University) is the centerpiece of my recent work in this line and the one occupying the bulk of my attention of late. Initially starting as an outgrowth of the first line (on the connections between cognitive neuroscience and sociology), the probabilism project has grown into a complete reconsideration of the intellectual trajectory of two major classical theorists, namely, Max Weber and Pierre Bourdieu. 

Initial papers outlining the argument for reconsidering the work of these two major theorists as belonging to a neglected tradition of ``sociological probabilism'' have been published in \textit{Sociological Theory} (\textbf{A159}) and \textit{Theory \& Society} (\textbf{A160}), prominent outlets that, we hope, will put the (initially counter-intuitive idea) out there. While impossible to summarize in a few words, the basic idea is that for both Weber and Bourdieu, ``probability'' was a resource to theorize social action and social order, not a method to compute odds or quantify chances. Both developed a unique and, heretofore, unappreciated way to think of \textit{Chance} as embedded in reality and as an objective possibility people are oriented to in their everyday courses of action. A full-length book manuscript, tentatively entitled \textit{Orienting to Chance}, is currently in the works (with a plan to submit it to Oxford University Press). The manuscript excavates the historical origins of this form of theoretical probabilism, its fall as a result of a history of misunderstandings and misappropriations while tracing its subterranean lineage via Weber and Bourdieu (drawing on the previously published work) while outlining its more general implications for sociological theory, sociological methodology, and substantive explanation. 

\subsubsection*{The ``Cultural Theory Essays'' Project}
This line of work deals with ``theorizing (the nature of) culture.'' So far, it has been developed as a series of papers---which will eventually turn into a book of essays---most of which are an outgrowth of a theory paper----already quite influential going by conventional metrics---published in \textit{American Sociological Review} (ASR) in 2017, distinguishing between different modalities of culture, particularly the difference between culture internalized in people and culture externalized in the world, and within people, culture that can be verbalized and culture that cannot. Since my last review, I have published several papers extending and revising the insights of the original ASR piece and developing its implications for various fields in the social and behavioral sciences. 

For example, a paper published in a special issue of \textit{Sociological Forum} dedicated to ``culture and cognition'' (\textbf{A151}) examines the neglected concept of internalization in cultural analysis, providing a refurbished conception of internalization processes that accommodate the various forms in which culture can be taken up by people, which I refer to as \textit{knowledge-that}, \textit{knowledge-how}, and \textit{knowledge-what}. Two recent papers in this line published in the \textit{Journal for the Theory of Social Behaviour} (\textbf{A148} and \textbf{A162}) examine the question of explaining action via internalized habits and the thorny question of the theoretical status of ``implicit culture.'' The first argues that habit-based explanation should be given priority in sociology over the usual explanation of action by pointing to intentions, beliefs, and desires while clarifying the multimodal nature of habit. The second paper argues for clarification and disaggregation of the idea of ``implicitness'' by differentiating \textit{implicit public culture} and \textit{implicit personal culture} (similar to the aforementioned 2017 \textit{ASR} piece), along with different flavors of implicitness at the personal level, namely, implicitness as automatic versus unconscious culture. Ongoing work in this line is a paper (\textbf{C42}) entitled ``Kinding Culture''---which is planned as the opening chapter in the book of essays----currently under review at the \textit{Journal for the Theory of Social Behavior}. The paper is the first to develop coherent guidelines for identifying cultural phenomena, separating them from other aspects of social behavior studied in different sciences. Lastly, a recent paper (\textbf{A169}) published in \textit{Philosophy of the Social Sciences} outlines a framework of how the multilevel conception of culture developed in the aforementioned work fits in with the recent movement to develop an ``analytical sociology,'' emphasizing explanation via nested socio-cultural mechanisms. The paper argues that there is indeed an apt fit between the two approaches and develops a vocabulary and analytic machinery for embedding cultural explanations in multi-level systems of social mechanisms. 

\subsection*{Social Network Analysis \& Network Dynamics}
The other major ongoing line of work further developed since my last review concerns a series of collaborative (closer to ``team based'' science) projects with a slew of colleagues at the University of Notre Dame, Rensselaer Polytechnic Institute, the City University of New York, and the Wroc\l{}aw University of Technology in Poland (including David Hachen, Boleslaw K. Szymanski, Hernan Makse, and Rados\l{}aw Michalski). The most recent highlight here is my winning (as co-PI in a collaborative proposal across three institutions) an NSF grant from the Human Networks and Data Science Division for \$999,000 to study the ``Dynamics and Mechanisms of Information Spread via Social Media,'' research work started this academic year).\footnote{\href{https://www.nsf.gov/awardsearch/showAward?AWD\_ID=2214216}{https://www.nsf.gov/awardsearch/showAward?AWD\_ID=2214216}}

Beyond that, the most recent work in this line consists of numerous papers published in high-impact outlets dedicated to general science, social networks, and computational social science. Highlights in this group include a paper developing a novel formal model of how cognitive traces of social contacts and social ties co-evolve in a dynamic communication network (\textbf{A145}), the cognitive side of the model, dubbed \textit{CogSnet} is inspired by work in the cognitive psychology of memory and is meant to capture how the memory trace of a social tie is re-activated with each act of communication and then decays gracefully until the next interaction. We fit the model to data I and co-PIs collected at Notre Dame (the \textit{NetSense} data) and find that it provides a superior fit compared to other dynamic network models. In \textbf{A163}, my collaborators and I extended the \textit{CogSNet} model to consider the ranking of social contacts across multiple dimensions of salience (``tie strength''). We fitted it to an even more expansive dynamic network data set I collaborators collected at Notre Dame (the \textit{NetHealth} data)\footnote{\href{https://sites.nd.edu/nethealth/}{https://sites.nd.edu/nethealth/}.}, once again demonstrating its superior fit compared to competitor models using the standard machine learning benchmarks.  Other work in this line is a highly accessed (in terms of downloads) paper (\textbf{A149}) on the dynamic evolution of groups and social attitudes in a natural ecological context published in \textit{Scientific Reports}. I wrote and conceptually developed the theoretical front end for all three of these papers, drawing on sociological work on social cognition, group, and network dynamics. One very recent paper in this line uses a simplified version of the \textit{CogSNet} model of network memory imprints (the Hawkes process) to predict changes in relationship labels (e.g., an acquaintance becoming a ``friend'') in dynamic networks (\textbf{A166}).

Finally, we have a series of papers on the co-evolution of social networks and fitness, sleep behaviors, physiological markers, and social attitudes, subjective loneliness, wearable use (\textbf{A144}, \textbf{A146}, \textbf{A147}, \textbf{A150}, \textbf{A157}, \textbf{A158}, \textbf{161}) that came from a unique data source collected as part of the NIH-funded NetHealth project at the University of Notre Dame, of which I was the PI. These include papers on the link between social networks and Fitbit usage, physiological markers, loneliness, physical activity, political attitudes, mental health, and sleep behaviors (among others).

\section*{Teaching}
Since my last review, I have taught one main class at the undergraduate level, Soc 111 (Social Networks, two times, once with an added honors section), and one main course at the graduate level, Soc 204 (Sociological Theorizing). I have also co-taught the required two-quarter first-year seminar introducing graduate students to sociology theory and research (202A, 202B) for three academic years and the seminar on social network analysis (208B)

\subsection*{Social Networks (111)}
Regarding the undergraduate class, the main challenge is to take a somewhat challenging (and sometimes abstract and math-heavy material) and convey it to students who may be intimidated by math. SOC 111 attracts mainly sociology majors looking to fulfill a methods requirement with minimal background in math, and the idea is to convey the material in a way they can understand and take away the core concepts in a way usable outside class. The Social Networks class is lecture-centric, so I have worked with TAs through the various iterations to make the slides as informative and organized as possible. I have also worked with repeat TAs (e.g., Carmella Stoddard) to ensure that coverage of examples and exercises is uniform across discussion sections. This approach helped improve student learning and engagement in the course, which now gets generally high marks from students. 

The remaining challenge (for me) is that no textbook resource is sufficiently centered on sociology and at the right level of ``hardness'' in terms of the mathematical concepts involved. Some of the textbooks at the right level of accessibility regarding math are only marginally related to Sociology, and the sociology-centric books are too basic. As a result, I have been developing an online textbook for the class, using freely available, open-source tools to write and publish it online.\footnote{\href{https://olizardo.github.io/networks-textbook/}{https://olizardo.github.io/networks-textbook/}} While this work is a ``work in progress,'' the material is already usable (and I have used it for the last two iterations). This has taken the course to a new level since there is maximum synergy between what students read and the lectures. Teaching the class has been a rewarding experience, primarily when I work more closely with students either in an honor's seminar setting (Soc 189).

\subsection*{Sociological Theorizing (204)}
The last version of this class I taught was completely revamped in terms of content and approach compared to previous iterations. I created a thematically unified set of readings that finally fit the quarter format and reorganized the seminar for more student-led discussions. I had mostly members of last year's first-year cohort in the class, and the result was a lively seminar that was unfortunately interrupted at the end by labor action at the end of last academic year. 

\subsection*{Theory and Research in Sociology (202A and 202B)}
Prof. Cecilia Menjivar and I designed the class as a substantive introduction to sociology and a professionalization tool. The class combines student-led discussion weeks featuring an introduction to both overall approaches to social analysis and specific areas of sociology, with students presenting draft literature reviews (Fall) and proposals (Winter). This setup worked well over the three years we taught the class, and getting to know the younger graduate cohorts, their interests, and early projects has been fun. 

\subsection*{Social Network Methods (208B)}
Despite the word ``methods'' in the title, I designed this class as a reading seminar to introduce graduate students to social network analysis as both a general perspective in sociology and a substantive field. We read representative work across various sub-areas of social network analysis published in mainstream sociological journals. The graduate students responded well to the readings and engaged in some of the best and most lively discussions I have taken part in a graduate seminar. So, overall, I would mark this as a successful entry.

\section*{Service}
\subsection*{Departmental}
I have served on various departmental committees since my last review. I have been a member of multiple ad hoc committees concerned with review, tenure, and promotion cases. In AY 2020-2021 I was a member of the Department's executive commmittee. Since 2021, I have served on the (work-intensive) merit review committee for two years in a row. During the review period, I have also served as a faculty reader for the sociology of culture (twice) and the economic sociology (once) exams. 

\subsection*{Graduate Student Training}
I am also heavily involved in graduate student training and mentorship in the department, currently playing either a lead or substantial advisory role for five sociology graduate students at various stages in the program (Nida Sanglimsuwan, Acton Feng, Anthony Shu, Aya Konishi), guiding them through the MA writing process and writing letters of support for various funding opportunities. During the review period, I served as a committee member in three successfully defended dissertations: Alina Arseniev-Koehler, Bernard Koch, and Anthony James Williams. I currently serve as a committee member of three yet-to-be-defended dissertations: Andrew Chalfoun, Carmella Stoddard, and Kaiting Zhou, regularly meeting with and advising each student. During the review period, I also directed Acton Feng's completed MA thesis and served as a second reader in three completed MAs: Kaitlyn Cunningham, Nida Sanglimsuwan, and Riley Ceperich. Furthermore, I wrote letters of support and provided job market, job talk, and negotiation mentoring/advice in the successful job market stints of Alina Arseniev-Kohler (Purdue Sociology by way of a two-year postdoc at UCSD) and Bernie Koch (Chicago Sociology). Finally, I also currently serve as the faculty advisor for the department's ``Culture Working Group.'' 

\subsection*{University}
I volunteer yearly to review applications for the Graduate Summer Research Mentorship and/or the Graduate Research Mentorship programs. Last academic year, I was also nominated by the department chair to be sociology's representative in the UCLA Regents Scholarship committee, a competitive program aimed at attracting academically excellent undergraduates to UCLA, where we review and rank dozens of applications from admitted seniors.

\subsection*{Disciplinary}
I have continued to be heavily involved in editorial work, having been elected this year to be one of the lead editors of \textit{Sociological Theory}, the American Sociological Association's main journal dedicated to theory. I sit on the editorial advisory board of various other sociology journals, including \textit{Poetics}, \textit{Social Forces}, and \textit{Theory \& Society}, and serve on the editorial board of two book series: The Culture \& Economic Life Series from Stanford University Press and the Cultural Sociology series from Princeton University Press. I also serve as a member of the steering committee member for \textit{SocArxiv}, a project designed to foster open science publication practices in sociology. For the last three years, I have been a  member and then chair of the committee for the American Sociological Association's (ASA) W.E.B. Du Bois Career of Distinguished Scholarship Award while helping organize and preside several sessions at the annual ASA meeting. Finally, during the review period, I served as an external member in the dissertation committees of three successfully defended dissertations: Alessandra Lembo (Chicago Sociology), Gordon Brett (Toronto Sociology), and Tae-Ung Choi (Kellogg, Management and Organizations). 

\subsection*{Interdisciplinary}
I continue to serve on the Board of Reviewing Editors for \textit{Science}, the top general science journal, advising the main editor for social and behavioral sciences. Additionally, I was selected as an Associate Editor of the new data science journal \textit{Discover Data}. I am also a regular Program Committee Member for various yearly conferences dedicated to Data Science, Network Science, Machine Learning, and related topics, including conferences on \textit{Advances in Social Networks Analysis and Mining}, \textit{Knowledge Discovery and Data Mining}, Social Influence, and Computational Social Science.

\section*{Diversity}
I have contributed to the university's and department's mission to foster and nurture diversity in various ways. As a nonwhite faculty member in a majority-white discipline (and departments throughout my career), I think this is an important and valuable goal. One of the things that attracted me to UCLA was the diversity of the undergraduate body across all dimensions. Accordingly, I have not hesitated to serve as a formal or informal advisor to first-generation students, students of color, and students from underrepresented backgrounds. This year, for instance, I was the faculty advisor for Jacarra Knowles, a graduating sociology senior who did a research project on the effect of online social media images on young Black women's body image. As noted earlier, I am developing a free textbook resource for the Social Networks class, which, I think, will help students from economically disadvantaged backgrounds get access to study materials, thus contributing to UCLA's commitment to equity and access.
\end{document}
 


