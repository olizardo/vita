
\ind {\bf Omar Lizardo} and Sara Skiles. ``After omnivorousness: Is Bourdieu still relevant?''  Pp. 90-103 in L. Hanquinet and M. Savage (Eds.), {\em Routledge International Handbook of the Sociology of Art and Culture}. Routledge. \href{https://www.routledgehandbooks.com/doi/10.4324/9780203740248.ch5}{\nolinkurl{doi:10.4324/9780203740248.ch5}}

\ind David Galehouse, Tommy Nguyen, Sameet Sreenivasan, {\bf Omar Lizardo}, Gyorgy Korniss and Boles\l{}aw K. Szyma\'{n}ski. ``Impact of network connectivity and agent commitment on spread of opinions in social networks.'' Pp. 149-160 in S. Schatz, J. Cohn and D. Nicholson (Eds.) {\em Advances in Cross-Cultural Decision Making}. CRC Press. 

\ind {\bf Omar Lizardo}. ``Embodied culture as procedure: Cognitive science and the link between subjective and objective culture.'' Pp. 70-86 in A. Warde and D. Southerton (Eds.), {\em The Habits of Consumption: COLLeGIUM: Studies Across Disciplines in the Humanities and Social Sciences}, Volume 12. Helsinki Collegium of Advanced Studies. \href{http://hdl.handle.net/10138/34223}{\nolinkurl{http://hdl.handle.net/10138/34223}}

\ind {\bf Omar Lizardo}. ``Jean Piaget: Sociology beyond holism and individualism.'' Pp. 315-322 in C. Edling and J. Rydgren (Eds.), {\em Sociological Insights of Great Thinkers: Sociology through Literature, Philosophy, and Science}. Praeger. 

\ind {\bf Omar Lizardo}. ``Culture and stratification.''  Pp. 305-315 in J. R. Hall, L. Grindstaff and M-C. Lo (Eds.), {\em Handbook of Cultural Sociology}. Routledge. \href{https://www.routledgehandbooks.com/doi/10.4324/9780203891377.ch29}{\nolinkurl{doi:10.4324/9780203891377.ch29}}

\ind Albert J. Bergesen and {\bf Omar Lizardo}. ``Terrorism and hegemonic decline.'' Pp. 227-240 in J. Friedman and C. Chase-Dunn (Eds.), {\em Hegemonic Decline: Present and Past}. Paradigm Publishers. \href{https://doi.org/10.4324/9781315634173}{\nolinkurl{doi:0.4324/9781315634173}} 

\ind Albert J. Bergesen and {\bf Omar Lizardo}. ``Terrorism and world system theory.'' Pp. 9-23 in Ryszard Stemplowski (Ed.), {\emph{Transnational Terrorism in the World System Perspective}}. The Polish Institute of International Affairs.\textcolor{uclablue}{$^{7}$}\footnotetext{\textcolor{uclablue}{$^{7}$}Translated (Polish) and reprinted as:  Albert J. Bergesen and {\bf Omar Lizardo}. 2002.  ``Terroryzm a teoria systemu swiatowego (world-system).'' {\em Polski Przeglad Dyplomatyczny} 2:  15-31.}