\documentclass[a4paper,11pt]{extarticle}
\usepackage[utf8]{inputenc}
\usepackage{geometry}
    \geometry{a4paper, margin = 0.82in}
\title{Self Statement}
\date{\vspace{-5ex}}

\usepackage[colorlinks=true, urlcolor=blue]{hyperref}

\begin{document}
\maketitle

\section*{Research}
Since my last review (Merit to Professor, Step IV o/s, Effective July 1, 2024), I have pursued active research agendas across a wide variety of areas. I have published nine articles in peer-reviewed journals, along with five book chapters in edited collections in my areas of expertise. I have also published a major book monograph with the University of Chicago Press (co-authored with my student Michael Strand) outlining a major new approach in social theory (\textbf{A184}), which we label \textit{probabilism}. I was also elected chair of the American Sociological Association's section on Mathematical Sociology. Accordingly, my scholarship and research accomplishments during this period showcase a multifaceted engagement with the interplay of culture, social networks, and cognition within sociology. 

\subsection*{Orienting to Chance}
\textbf{A184} presents a provocative re-evaluation of probability within social theory, arguing for a shift from a dominant frequentist interpretation to a revitalized         \textit{probabilism}. The central claim is that probability is not merely an epistemological tool for calculating frequencies, but an objective feature of the world itself, deeply embedded in social action and cognition. To establish this new idea, we take a genealogical approach, tracing the historical entrenchment of frequentism, critiquing its limitations, and excavating an alternative tradition of objective probability, demonstrating its deep implications for understanding social phenomena. 

Central to our argument is the reintroduction and redefinition of \textit{Chance} as a form of ``objective probability." Drawing from the work of Max Weber, who in turn drew from his contemporaneous polymath Johannes von Kries, \textit{Chance} refers to a specific kind of objective (but not necessarily calculable) probability, intrinsically linked to expectation and embedded in institutional arrangements we refer to as probabilistic orders. We argue that the concept of \textit{Chance} is integral for understanding Weber's sociology, and plays a role in W.E.B. Du Bois's call for a sociology that unhesitatingly addresses ``Law" and ``Chance" at the same time. Objective probability is distinct from understandings of this term in both contemporary frequentism and Bayesianism, in that probability is understood as residing in the world itself, revealing itself to actors as ``objective possibility" rather than stemming solely from human ignorance or subjective belief. This framework provides a robust alternative to purely epistemological views of probability, asserting its relevance as a tool for theorizing action and social order. 

We argue that probabilism offers a new paradigm that transcends the long-standing debates between realism and interpretivism in social theory by focusing on probabilistic orders, adequate causation, and \textit{Chance} as fundamental analytical units, thus transcending standard structure-agency dualism frameworks. In this respect, we critique the ``central problems" style of theorizing dominant in contemporary theory, of which the structure-agency dualism is emblematic, by reframing it as dealing with ``\textit{Chance}/expectation loops," emphasizing that social actors possess a practical understanding of probabilities that shapes their actions and choices within a range of possibilities. Drawing inspiration from Bruno Latour's ``clean slate" approach to studying associations, we propose moving beyond macro-level concepts to focus on objective probability as a basic unit of analysis. We also introduce the notion of a ``test'' as a mechanism that actively generates \textit{Chance}, creating and maintaining probability orders. This leads to new tasks for social theory, including translating observations into probability orders, designing new tests, and modeling collective objects as these orders. 

\textbf{A184} also integrates insights from Predictive Processing (PP) in cognitive science, presenting it as a model for how prediction (and thus probability) is fundamental to perception, cognition, and action. PP's core idea of ``prediction error minimization" (PEM) and ``active inference" is aligned with the sociological ``Chance/expectation loop," where the brain constantly adjusts its generative models of the environment's probabilistic structure. This framework argues for a ``radical continuity" between the personal and subpersonal domains, and the horizontal and vertical levels of explanation in theorizing cognition, asserting that sociocultural practices actively shape these predictive processes and that, ultimately, the study of action is equivalent to the study of probability, proposing the idea of ``looping effects" (interpretation, description, and probability modulations) as mechanisms that create structures and probability orders.

Overall, \textbf{A184} calls for a fundamental recalibration of sociological inquiry, advocating for a ``probabilistic truth" that can be compelling without being reducible to mere statistics. It challenges sociologists to redefine their methods to uncover ``where the action is" and ``what is the \textit{Chance}?" in social life. The book also delves into how optimism, faith, and promises are deeply intertwined with the inherent ``looping" of expectation and \textit{Chance}, and how algorithmic power can manipulate these processes to control prediction and action, thus giving rise to new forms of power that operate through ``immanent matching" and the shaping of ``fate". Ultimately, \textbf{A184} is a call for a more integrated and ontologically grounded social theory, capable of bridging disciplinary divides and providing a more comprehensive understanding of the social world by re-centering probability as a core analytical concept.

\subsection*{Empirical Work in Social Network Analysis}

\textbf{A179} is the main study from an NSF-funded project on the spread of misinformation in social media. This study revises the traditional ``two-step flow communication model," proposing a more complex ``multi-step and multi-actor model" to better capture information dissemination in the social media era. The conventional model, designed for the legacy media era, posited a linear flow from sources to the public mediated by ``opinion leaders.'' However, the research finds that contemporary information spread involves direct access from adopters to sources (one-step flows), longer multi-step flows, and significant ``horizontal information flow" among adopters. A key finding is the emergence of two new central figures: ``influencers" (who build authority through online presence and engaging content) and ``opinion-leading influencers" (traditional opinion leaders who adapt an influencer strategy). The analysis, based on 2020 US presidential election Twitter data, reveals that in the second step of diffusion, opinion-leading influencers and influencers wield strong influence, while traditional opinion leaders account for only a small fraction of the overall information flow. Adopters also function as significant mediators for a good chunk of downstream information flow. Intriguing differences are also observed across ideological lines: right-leaning information diffusion shows a greater footprint from social-media-based influencers at earlier steps compared to left-leaning sources, where adopters play a larger role in horizontal transmission. We conclude that while intermediation persists, its landscape is diversified, requiring models that account for heterogeneous actors and multi-pathway diffusion.

\textbf{A173} is a substantive paper developing an approach to the analysis of the coupling of two-mode and one-mode networks, which we term ``dynamic focus theory.'' This framework is a reconceptualization of classical focus theory in social network analysis, designed to account for the co-evolution of multiple network layers. While classical focus theory primarily conceptualizes social foci (groups, organizations, settings) as ``tie generators"---bringing people together with similar interests and backgrounds to form and strengthen ties---dynamic focus theory expands this by recognizing that foci can also act as ``diffusion points" where tastes, activities, and affiliations spread among people via interpersonal ties. The study empirically investigates these dual roles across different types of foci (musical genres, club types, course areas, and activities) using longitudinal data and Stochastic Actor-Oriented Models (SAOMs) from the \textit{NetSense} and \textit{NetHealth} studies, which I led and collected at the University of Notre Dame. The findings indicate significant variation across foci: musical genres function primarily as ``taste-diffusers," club types exclusively as ``tie generators," and course areas and activities exhibit both ``tie generator" and ``taste diffuser" effects. This analysis highlights how the properties of social foci influence the strength of these dynamic effects and underscores the importance of considering shared affiliations when analyzing changes in interpersonal networks. Overall, dynamic focus theory offers a more nuanced understanding of how social networks and cultural contexts are dynamically coupled, bridging selection and influence processes.

\textbf{A172} is a more methodological piece introducing a simplified, non-iterative method for calculating generalized relational similarities in two-mode network data, offering an alternative to an established iterative approach. The proposed method, termed ``two-mode relational similarities," directly leverages the classical dual projection approach to obtain similarity matrices between actors (based on shared objects) and between objects (based on shared actors). This approach adheres to established principles of generalized equivalence (actors are similar if they have similar relations to objects, and vice versa) and duality (actors are similar if they relate to similar objects, and objects are similar if they relate to similar actors). Through empirical re-analyses of the Southern Women's data and 112th U.S. Senate voting data, the paper shows that this non-iterative method yields results substantively indistinguishable from existing strategies, often providing slightly better or comparable clusterings without relying on arbitrary convergence criteria. 

\textbf{A180} investigates the multifaceted influences of social, personal, physiological, environmental, and behavioral factors on sleep and active heart rate dynamics among college students. Utilizing longitudinal data from the \textit{NetHealth} study collected from 487 participants over 637 days via wearable technology, we use latent growth-curve modeling to unravel complex relationships impacting heart rate variations. The study highlights the importance of distinguishing heart rate patterns during sleep from those during wakefulness, as both reflect different physiological responses and carry distinct health implications. We show that the average number of in-study friends and daily network size positively affect both sleep and active heart rates, suggesting that network activity is systematically linked to physical activity. The research enriches the understanding of cardiovascular health and suggests implications for targeted healthcare interventions and future research.

\subsection*{Theorizing Across Disciplinary Boundaries}

\textbf{A174} contributes to the conceptual clarification of the central notion of ``social tie" in social network analysis by employing Barsalou-frames, a technique borrowed from cognitive psychology and linguistics. Barsalou-frames represent concepts as recursive attribute-value structures, which can graphically depict the hierarchical organization of attributes like ``tie strength" (composed of sub-attributes such as support, closeness, duration, and frequency of interaction). This frame-based representation clarifies how typologies (e.g., "strong ties" and ``weak ties") are implied by the frame's structure and helps in specifying intra-theoretical conceptual relations, such as how different theories impose constraints on attribute values. Furthermore, the frame approach illuminates inter-theoretical commonalities and differences for frameworks that use the idea of social ties as their main object. The frame approac  also helps identify and avoid ``construct confounding" in network theory, namely, the erroneous treatment of analytically distinct tie attributes as sub-attributes of another unrelated concept, which can lead to conceptual confusion and invalid empirical inferences. By making conceptual structures explicit, frames support theoretical development, theory testing, and a more rigorous approach to defining and classifying social ties.

\textbf{A177} enriches sociological theories of motivation by integrating insights from affective and cognitive neuroscience, particularly distinguishing between ``wanting" and ``liking" as distinct affective mechanisms. Sociological models of motivation have traditionally focused on ``pain avoidance" or re-establishing balance from discomfort (control models) as the main source of motivation, often neglecting the ``pleasure-seeking" aspect. Cultural sociology has introduced "goal-directedness" but needs further clarification on the motivational process. Drawing on Jaak Panksepp's work on primary affective systems, we highlight that motivation is not generic but tied to specific affective systems. ``Wanting" is conceptualized as the anticipatory, desire-driven pursuit of goals, underpinned by dopaminergic SEEKING systems, characterized by ``intense interest" and ``eager anticipation," which energizes long-term goal pursuit even through unpleasurable activities. In contrast, "liking" refers to the experience of pleasure and satisfaction upon obtaining a desired object, mediated by opioid-releasing hedonic ``pleasure spots" in the brain, and is often associated with the intrinsic enjoyment of aesthetics and cultural consumption. This paper thus offers a more proactive, desire-driven model of the social agent, moving beyond prevailing sociological models that emphasize reactive actors responding to incongruence. 

\subsection*{Studies Cultural and Cognitive Sociology}

\textbf{A175} introduces and examines a novel metaphor system for conceptualizing states in conceptual metaphor theory: ``states are physical qualities," arguing it is an elaboration of the more abstract ``states are locations" conceptual metaphor. Unlike ``states are locations," which primarily affords topological and temporal understandings, ``states are physical qualities" provides specific evaluative and practical implications, often situating states within a teleological framework of directed change. We focus on \textit{processing} as a culturally pervasive source domain for changes in physical qualities, with ``cooking" serving as the prototypical human intervention that transforms natural (unprocessed) objects into modified (processed) ones. This leads to two key metaphorical instantiations in social evaluation: ``authentic is unprocessed" (or ``inauthentic is processed") and ``developed is processed" (or ``undeveloped is unprocessed"). The ``authentic is unprocessed" metaphor assigns negative valence to processing, emphasizing an either-or, discontinuous judgment structure, often seen in aesthetic and moral judgments where any human intervention ``ruins" the object's "truth" or genuineness. Conversely, the ``developed is processed" metaphor places a positive valence on processing, conceptualizing it as continuous improvement towards a desired state. The paper argues that these metaphors, grounded in embodied experiences with physical objects, help people evaluate the utility and quality of individuals, actions, and objects.

\textbf{A176} addresses the ``macrogenre problem" in the sociology of taste, which questions the validity and utility of broad, vague genre labels used in standard surveys. It proposes a ``link-clustering approach" from network science to discover more focused ```microgenres" by exploiting the relational patterns within two-mode (person-by-genre) survey data. This method takes the person-to-genre link as the unit of analysis, allowing genres to be assigned to multiple overlapping clusters and revealing internal differentiation within macrogenres, thereby tackling the issues of overlapping categories and hidden heterogeneity simultaneously. Using this approach, the paper conducts two case studies on ``Heavy Metal" and ``Latin/Salsa" music, showing how link clustering uncovers audience segmentations and socio-demographic patterns that are obscured by macrogenre analysis. For instance, it reveals a version of Metal appealing to young, highly educated, racially and gender-neutral audiences, contrasting with traditional Metal demographics. Similarly, for Salsa, it identifies a high-cultural-capital microgenre preferred by older, more educated respondents, consistent with hierarchical taste differentiation. The approach is generalizable beyond musical tastes to other belief, opinion, and attitude data, enabling more nuanced sociological insights without requiring new data collection methods.

\subsection*{State of the Art Reviews of the Culture and Social Networks Area}
In \textbf{A171} I provide a framework for a unified theoretical foundation in the field of culture and network studies, proposing that the field is organized around two core ``theoretical imageries'': \textit{constructural theory} and \textit{cultural holes}. The constructural imagery, primarily rooted in the computational social scientist Kathleen Carley's model, posits that agents exchange cultural information when they interact, leading to increased cultural similarity, which in turn boosts the chance of future interaction, creating a positive feedback loop connecting cultural exchange to tie strength. This imagery has evolved through models like Network Ecology Models that explain how homophily leads to taste clustering in ``cultural niches", and my own Culture Conversion Model (CCM), demonstrating how cultural tastes drive the formation of network ties (developed in \textbf{A9}). The cultural holes imagery focuses on linkages between people and other social entities in two-mode networks to define positions in ``sociocultural" and ``culturosocial spaces". This framework can be used to understand phenomena like cultural omnivorousness as bridging cultural holes, the ``categorical imperative" in market settings, and the distinction between cultural and structural hole bridging in creativity and success. Research in this area also extends to culture-centric analyses, such as studying literary authors' roles in bridging cultural holes based on co-borrowing patterns, and measuring ``discursive holes" in text similarity networks.

\textbf{A178} reviews the theoretical and empirical literature linking social capital (resources embedded in networks) and cultural capital (tastes and habitualized cultural practices), inspired by Bourdieu's idea of the ``interconvertibility" of the different forms of capital. The chapter outlines two primary conceptualizations of social capital: resource \textit{heterogeneity} (diversity of positions known) and resource \textit{richness} (concentration of contacts in prestigious positions). Similarly, cultural capital is viewed as either \textit{diversity} in tastes (omnivorousness) or the consumption of high-status cultural goods. The chapter first examines how social capital converts into cultural capital, particularly through the Network Variety Model (NVM), which posits that diverse network connections lead to diverse tastes and cultural consumption. Conversely, the chapter explores how cultural capital creates social capital, particularly via my own Culture Conversion Model (CCM), which argues that cultural resources, especially the ability to consume certain forms of culture, lead to the formation and maintenance of social network ties and my own Cultural Matching Model (CMM) which further posits that cultural tastes affect networks by serving as a basis for homophily. Recent studies have provided empirical support for these conversion mechanisms, showing how shared tastes influence friendships and how cultural profiles link to socioeconomic network resources and the activation of social ties for job searches.

\textbf{A181} reviews the interdependent relationship between cultural tastes and ``everyday networks," identifying two dominant approaches in the field, similar to those dealt with in \textbf{A178}. The first perspective views networks as shapers of tastes, where network ties and their structure (e.g., homophily) are primary causal drivers.  The second perspective, tastes as shapers of networks, posits that cultural aptitudes drive tie formation, maintenance, and decay processes. Both perspectives are consistent with a ``constructural imagery" of the mutual constitution between taste and networks where influence, imitation, conversion, and matching co-occur. However, longitudinal data are crucial for adjudicating the predominant mechanisms. The chapter also highlights the need for more sophisticated measurement of cultural tastes to match network measurement advancements.

\textbf{A183} critically reviews the literature on culture and entrepreneurship, arguing that it largely relies on an outdated and theoretically limiting ``shared values" conception of culture, often operationalized using country-level averages of constructs like individualism-collectivism. We contend that this approach suffers from the ecological fallacy, ethnocentric biases towards ``WEIRD" countries, and a lack of temporal robustness in empirical findings. Crucially, empirical evidence shows that within-country variance of values is often much larger than between-country variance, challenging the notion of ``shared meaning" as a defining characteristic of culture. Instead, we advocate for a ``disaggregated, multilevel, and heterogeneous conception of culture" that distinguishes between ``public culture" (external, environmental elements like codes, frames, narratives) and ``personal culture" (internalized beliefs, skills, dispositions), and further differentiates between ``declarative" (explicit knowledge) and ``non-declarative" (tacit, habitual) forms of personal culture (as developed in \textbf{A111}). The chapter concludes by emphasizing that adopting this multilevel, heterogeneous view and diversifying measurement methods will allow for a more robust understanding of how culture influences entrepreneurship, moving beyond shared-value frameworks and enabling the study of diverse contexts and mechanisms.
\end{document}
 


